\documentclass[10pt, a4paper]{article}
\usepackage[paper=a4paper, left=1.5cm, right=1.5cm, bottom=1.5cm, top=3.5cm]{geometry}
\usepackage[latin1]{inputenc}
\usepackage[T1]{fontenc}
\usepackage[spanish]{babel}
\usepackage{indentfirst}
\usepackage{fancyhdr}
\usepackage{latexsym}
\usepackage{lastpage}
\usepackage{graphicx}
\usepackage{caption}
\usepackage{subfigure}
\usepackage{caratula}
\usepackage{listings}
\usepackage{mathtools}
\usepackage[colorlinks=true, linkcolor=blue]{hyperref}
\usepackage{calc}
\usepackage{cite}
\usepackage[nottoc,numbib]{tocbibind}

\sloppy

\parskip=5pt % 10pt es el tama�o de fuente

% Pongo en 0 la distancia extra entre �temes.
\let\olditemize\itemize
\def\itemize{\olditemize\itemsep=0pt}

% Acomodo fancyhdr.
\pagestyle{fancy}
\thispagestyle{fancy}
\addtolength{\headheight}{1pt}
\lhead{Algoritmos y Estructuras de Datos III}
\rhead{$1^{\mathrm{er}}$ cuatrimestre de 2014}
\cfoot{\thepage /\pageref{LastPage}}
\renewcommand{\footrulewidth}{0.4pt}

\author{Algoritmos y Estructuras de Datos III, DC, UBA.}
\date{}
\title{Trabajo pr\'actico 1}

\begin{document}

\titulo{Trabajo Practico 1}
\submateria{Primer cuatrimestre 2014}
\materia{Algoritmos y Estructura de Datos III}
\grupo{Grupo 1}
\integrante{Blundi, Solange}{336/10}{solange.blundi@gmail.com}
\integrante{Paez, Ariel}{668/09}{twizt.hl@gmail.com}
\integrante{Guerson, Matias Carlos}{925/10}{matias.guerson@gmail.com}
\integrante{Inzaghi Pronesti, Patricio Ezequiel}{255/11}{pinzaghi@dc.uba.ar}
\maketitle

\newpage

\tableofcontents

\newpage
\section{Problema 1: Camiones sospechosos}
\subsection{Descripci\'on del problema}

En este ejercicio se plantea la necesidad de inspeccionar los camiones de una determinada empresa; para lo cual se puede contratar a un inspector durante un intervalo de d\'ias D. \\ 
Dados el intervalo y la cantidad de camiones c de dicha empresa, junto con la lista de d\'ias en que va a pasar cada cami\'on por el lugar donde estar\'a el inspector, la idea es elegir el d\'ia inicial de contrataci\'on tal que se maximice la cantidad de camiones inspeccionados. En caso de haber varios d\'ias iniciales que permitan controlar la misma cantidad de camiones (siempre y cuando esta cantidad sea m\'axima), cualquiera de estas opciones ser\'a correcta. \\

Por ejemplo, suponiendo que podemos contratar al inspector por 3 d\'ias consecutivos y dada una lista de d\'ias en que pasan los camiones L=[1,3,4,5,7,8,9] se puede ver en el siguiente diagrama que tanto el d\'ia 3 como el 7 son posibles respuestas.\\

\begin{figure}[h]
\begin{center}
\includegraphics[scale=0.7]{./img/ej1_intervalo.png}
\caption{Ejemplo de intervalos optimos}
\end{center}
\end{figure}

\subsection{Resoluci\'on}

Para resolver el problema partimos de la idea de recorrer cada elemento de la lista de d\'ias en que pasan camiones, es decir cada posible d\'ia inicial, y a partir de ah\'i sumar uno por cada d\'ia con camiones que entrara en el rango de contrataci\'on.\\
Sin embargo, como ese algoritmo no cumple la complejidad requerida (estrictamente menor que O($n^2$)) decidimos mejorarlo reemplazando la segunda iteraci\'on sobre la lista obteniendo el \'indice del primer d\'ia con camiones fuera del rango de contrataci\'on (en $O(log(n))$) y rest\'andoselo al \'indice del primer d\'ia de contrataci\'on. De esa forma sabemos cu\'antos camiones ser\'ian inspeccionados en ese per\'iodo.\\\
As\'i, llegamos a un algoritmo que hace lo siguiente:
\begin{itemize}
\item poner m\'aximo = 0
\item poner i = 0
\item mientras i<long(listaD\'ias)
\begin{itemize}
	\item poner ultimoD\'ia = listaD\'ias[i] + rangoContrataci\'on - 1
	\item poner primeroFueraRango = indice del primer elemento mayor a ultimoD\'ia
	\item poner inspeccionados = primeroFueraRango - i
	\item si inspeccionados > m\'aximo entonces poner m\'aximo = inspeccionados
	\item poner i = i + 1
\end{itemize}
\item devolver m\'aximo
\end{itemize}

En nuestra implementaci\'on la forma de obtener el $indice del primer elemento mayor a ultimoDia$ es utilizando el m\'etodo upperBound, que devuelve un iterador al primer elemento mayor al indicado. Con eso calculamos el \'indice de dicho iterador y lo usamos luego para restarlo con el \'indice actual y saber cu\'antos elementos hay entre ambos l\'imites.\\

Como tipo de datos elegimos:
\begin{itemize}
\item Arreglo: para la lista de d\'ias. Este tipo de datos nos permite: 
\begin{itemize}
 \item ordenar en O(n*log(n))
 \item buscar primer elemento mayor en O(log(n)+1)
 \item y nos da la ventaja extra de los \'indices, que nos permiten saber c\'uantos elementos hay dentro de un rango con una simple resta de $fin$ menos $inicio$.
\end{itemize}
\end{itemize}

\subsection{Demostraci\'on de la resoluci\'on}

Para ver que el algoritmo propuesto resuelve el problema de obtener el m\'aximo de camiones inspeccionados vamos a abstraernos del problema concreto y vamos a plantearlo de la siguiente manera: dada una lista de enteros positivos y un offset, queremos encontrar, la sublista m\'as larga de elementos tal que est\'en entre x (que puede ser cualquier entero positivo) y x + offset - 1 inclusive.\\

Llamemos n a la cantidad de elementos de la lista y o al offset.\\
Antes de empezar ordenamos la lista, ya que inicialmente podr\'ia estar desordenada; as\'i que a partir de ahora vamos a asumir que tenemos una lista ordenada de manera creciente de enteros positivos. \\ \\

Para comenzar notemos que, de todo el universo de posibles intervalos, consideraremos s\'olo los que comienzan con cada elemento de la lista; es decir que si nuestra lista es [1, 3, 5, 6] y nuestro rango es 2, s\'olo vamos a mirar los intervalos [1,2], [3,4], [5,6] y [6, 7].\\ \\

Demostremos entonces que con los intervalos que estamos considerando alcanza ya que entre ellos hay al menos uno que tiene la sublista m\'as larga.\\
Dados un offset $off$ y una lista ordenada de manera creciente $l$ y sea $S$ el subconjunto de intervalos que vamos a considerar, expresada como: \\
\begin{center}
$ S = \left\lbrace s_i = \left[ e \rightarrow l | l_i \leq e \leq l_i + off - 1 \right] \right\rbrace \forall 1 \leq i \leq long(l) $ 
\end{center}
Es decir, es un conjunto de listas tal que todos los elementos de cada lista $s_i$ pertenecen a $l$ y est\'an dentro del rango $l_i$-$l_i + off - 1$. \\
Supongamos ahora que existe una lista ordenada $s'$ que cumple con lo pedido y que es \'optima.\\
Supongamos adem\'as que esta lista \'optima no est\'a en S; es decir que no est\'a siendo considerada en nuestra resoluci\'on.\\
Sabemos que, como la lista es finita y los n\'umeros que pueden entrar en los intervalos est\'an ah\'i, entonces \\
\begin{center}
($\forall$ elem $\in$ s') elem $\in$ l, \\
\end{center}
y en particular, \\
\begin{center}
$s'_1$ $\in$ l. \\
\end{center}

Adem\'as, como la lista est\'a ordenada y cumple con la condici\'on de que todos sus elementos est\'an dentro de un rango determinado por $inicio$-$inicio + o - 1$\\
\begin{center}
($\forall$ elem $\in$ s') $s'_1 \leq$ elem $\leq inicio + off - 1$.\\
\end{center}

Notemos ahora que $inicio$ $\leq$ $s'_1$, es decir que \\ 
\begin{center}
$inicio$ $\leq$ $s'_1$ $\Leftrightarrow$ $inicio + off - 1$ $\leq$ $s'_1 + off - 1$
\end{center}
Porque, recordemos, se trata de enteros positivos, con lo cual vale la desigualdad.\\

Miremos ambos casos:
\begin{itemize}
\item inicio + off = $s'_1$ + off. En este caso pertenecer\'ia a S, ya que inicio = $s'_1$ y $s'_1$ es un elemento de l.
\item inicio + off - 1 < $s'_1$ + off - 1. En este caso el per\'iodo comienza antes que $s'_1$, pero el primer elemento de l que entra en el rango es $s'_1$, con lo cual ninguna de las listas de S se est\'a perdiendo elementos previos a $s'_1$. Por otro lado, llamemos $s_i$ a la lista de S que comienza con $s'_1$; como $inicio + off - 1$ < $s'_1$ + off - 1, podemos deducir que a $\leq$ $ s_i$. Ac\'a llegamos a una contradicci\'on: no puede haber una lista \'optima que no est\'e en S.
\end{itemize}

Una vez probado lo anterior, es trivial saber que comparando uno a uno los tama\~nos de las listas de S vamos a encontrar cu\'al es \'optima.\\

\subsection{Complejidad del algoritmo}

Analizaremos a continuaci\'on la complejidad del algoritmo propuesto utilizando un pseudoc\'odigo simplificado como gu\'ia.

\begin{itemize}
\item ordenar listaD\'ias, $O(n*log(n))$
\item para cada d\'ia en listaD\'ias, $O(n)$
\begin{itemize}
	\item poner ultimoD\'ia = d\'ia + rangoContrataci\'on - 1
	\item poner primeroFueraRango = buscar(ultimoD\'ia, listaD\'ias), $O(log(n))$
	\item poner inspeccionados = primeroFueraRango - i
	\item si inspeccionados es mayor que m\'aximo poner m\'aximo = inspeccionados
\end{itemize}
\end{itemize}

Decimos que ordenar la lista toma O(n*log(n)) ya que en nuestra implementaci\'on usamos std::sort.\cite{sort}\\
Lo mismo pasa cuando decimos que encontrar el primero fuera de rango toma $O(log(n))$, para lo cual usamos std::upper\_bound.\cite{upper} \\

\newpage 

\subsection{Codigo fuente}

\lstset{language=C++,
                basicstyle=\ttfamily\footnotesize,
                keywordstyle=\color{blue}\ttfamily,
                stringstyle=\color{red}\ttfamily,
                commentstyle=\color{green}\ttfamily,
                morecomment=[l][\color{magenta}]{\#},
                breaklines=true
}
\begin{lstlisting}

typedef std::vector<int> LCamiones;
typedef pair<int, int> intervalo;

intervalo resolver(LCamiones& c, int periodo){
	
	int inicio = 1;
	int maxInspec = 0;
	intervalo resultado = intervalo();
	//vector<int>::iterator itCamiones;


	// Ordeno la lista de camiones en O(n*log(n))
	// http://www.cplusplus.com/reference/list/list/sort/
	sort(c.begin(), c.end());

	vector<int>::iterator ultimoCamion;
	int inspecTemp = 0;
	int cantCamiones = int(c.size());
	int finContrato;
	int ultimoVisto = 0;
	int index;

	// Recorro los dias en que pasan camiones
	for(int i=0;i<cantCamiones;i++){
		
		// me fijo si el ultimo que verifique no es igual al actual en caso de que hayan repetidos
		if(ultimoVisto != c.at(i)){
			// Primer dia fuera del rango
			finContrato = c.at(i) + periodo - 1;
			cout << "fin contrato: " << finContrato << endl;
			
			// Obtengo el primer camion fuera del rango O(log2(N)+1) donde N es la distancia entre inicio y final
			// http://www.cplusplus.com/reference/algorithm/upper_bound/
			ultimoCamion = upper_bound(c.begin(), c.end(), finContrato);
			cout << "primero fuera de rango: " << *ultimoCamion << endl;
			
			index = ultimoCamion - c.begin();
			inspecTemp = index - i;
			cout << "camiones: " << inspecTemp << endl;

			// Si encontre un inicio mejor(o igual) reemplazo el anterior
			if(inspecTemp >= maxInspec){
				maxInspec = inspecTemp;
				inicio = i;
			}
			inspecTemp = 0;
		}
		ultimoVisto = c.at(i);

	}

	resultado.first = c.at(inicio);
	resultado.second = maxInspec;
	
	return resultado;
}

\end{lstlisting}

\subsection{Casos de prueba}

Como casos de prueba para el algoritmo elegimos inputs de los siguientes tipos:

\begin{itemize}
\item Caso en el que el per\'iodo de contrataci\'on es mayor al del rango en que pasan los camiones. El d\'ia inicial debe ser el d\'ia en que pasa el primer cami\'on y la cantidad de camiones inspeccionados es igual a la cantidad de camiones.
\item Caso en el todos los camiones pasan el mismo d\'ia. El d\'ia inicial debe ser el d\'ia en que pasan los camiones y la cantidad es igual a la cantidad de camiones.
\item Caso en el que pasa un solo cami\'on por per\'iodo posible. El d\'ia inicial debe ser el d\'ia en que pasa el \'ultimo cami\'on y ese debe ser el \'unico inspeccionado.
\item Otros casos sin ninguna particularidadd a mencionar.
\end{itemize}	

Se pueden correr los mismos casos y verificar que funcionan en el directorio /ej1 dentro del trabajo entregado, ejecutando ./ej1 < prueba.txt.

\subsection{Performance}

Para realizar el testeo de performance creamos test.cpp, el cual genera casos de test pseudo-aleatorios, determinados por una semilla, la cual se pasa como parametro al ejecutable 'test', por ejemplo: ./test 5.

Este crea 100 casos de prueba con, como dijimos, una cantidad de d\'ias en la que pasar\'a el inspector entre 1 y 100.000, un n\'umero de camiones entre 1 y 10.000.000, y el d\'ia en el que pasar\'a cada cami\'on entre 1 y 10.000.000, todos pseudo-aleatorios.

En este caso no se evalualon casos bordes porque el algoritmo tiene como complejidad del peor caso limitado por el algoritmo de ordenamiento que se utiliza sobre los d\'ias que pasan los camiones que es de O(n*log(n)) donde n es la cantidad total de camiones.
Luego se busca n veces el primer cami\'on fuera del rango de d\'ias de contrataci\'on del inspector y esto tiene una complejidad acotada por O(log(n)+1) donde n es la cantidad de d\'ias de contrataci\'on.

\begin{center}
\begin{figure}[h!]
\includegraphics[scale=0.4]{./img/ej1_chart.png}
\caption{Tiempo transcurrido por eleccion de periodo de contratacion del inspector}
\end{figure}
\end{center}

Como muestra el gr\'afico, el algoritmo presenta tiempos de ejecuci\'on mucho mas \'optimos que la cota propuesta de $O(n*log(n))$.


\newpage
\section{Problema 2: La joya del Rio de la Plata}
\subsection{Descripci\'on del problema}

En este punto se nos pide resolver el problema de ubicar centrales de gas de manera estrat\'egica entre los pueblos de una cierta regi\'on.\\

Dados $n$ pueblos con sus respectivas coordenadas en el plano y $k$ centrales de gas, la idea es decidir en qu\'e pueblos ubicar las centrales. Adem\'as es posible construir tuber\'ias entre pueblos: una tuber\'ia que va de un pueblo con central a uno sin no s\'olo transporta el gas a este \'ultimo sino que adem\'as lo convierte en potencial proveedor. Es decir que se da una especie de transitividad entre los pueblos: si un pueblo $a$ con central de gas se conecta a otro pueblo $b$ y a su vez $b$ se conecta con $c$ entonces $c$ tambi\'en recibe gas (y puede distribuirlo).\\

El \'unico problema que presentan las tuber\'ias es que, a mayor longitud aumenta la posibilidad de rotura de las mismas. Obviamente es una situaci\'on que se busca evitar, por lo cual se nos pide que, al elegir los pueblos donde instalar las centrales y las conecciones entre los mismos minimicemos el tama\~no de la tuber\'ia m\'as larga.\\

Veamos algunos ejemplos. La siguiente es una situaci\'on trivial en la que tenemos la misma cantidad de pueblos que de centrales:

\begin{figure}[h]
\begin{center}
\includegraphics[scale=0.7]{./img/ej2_explicacion1.png}
\caption{Caso trivial}
\end{center}
\end{figure}

Los circulos colorados indican d\'onde fueron instaladas las centrales y, en este caso, no hay ninguna tuber\'ia.\\

Veamos el un caso m\'as complejo: 

\begin{figure}[h]
\begin{center}
\includegraphics[scale=0.7]{./img/ej2_explicacion2.png}
\caption{Caso con K = 3 y N = 6}
\end{center}
\end{figure}

Como dato, sabemos que $e_2 < e_3 < e_1$ con lo cual nuestro largo m\'aximo de tuber\'ia es $e_1$

\subsection{Resoluci\'on}

Para la resoluci\'on del problema decidimos primero modelarlo con grafos, de forma que los nodos representen a los pueblos, y las aristas (con pesos) las posibles conexiones entre pueblos. Los pesos est\'an dados por las distancias entre cada par de pueblos. \\

Como se nos pide minimizar el riesgo de rotura de las tuber\'ias a instalar, lo ideal ser\'ia lograr un esquema de conexiones tal que el largo de la tuber\'ia m\'as larga sea m\'inimo. Como veremos m\'as adelante, la ubicaci\'on de las centrales de gas no es de gran importancia ya que la transimisi\'on del servicio de gas se da por transitividad entre pueblos conectados (mientras alguno de ellos tenga una central). Por lo tanto lo importante es tener grupos de pueblos conectados a una misma central de manera que las tuber\'as necesarias sean de tama\~no m\'inimo. \\

Para obtener el esquema de conexiones antes mencionado partimos de pensar la regi\'on como un grafo completo; es decir que suponemos que todas las ciudades est\'an conectadas entre s\'i. Luego, a partir del grafo completo la idea es obtener un \'arbol generador m\'inimo del mismo utilizando el algoritmo de Prim (que es el que, con la implementaci\'on elegida nos permite alcanzar la complejidad requerida).\\
Finalmente cortamos las $k - 1$ aristas m\'as largas del \'arbol, obteniendo un grafo de $k$ componentes conexas tal que la arista m\'as larga del grafo es de peso m\'inimo.\\ 
Como se dijo antes, es indistinto cu\'al de los pueblos de cada componente conexa tiene instalada la central. Sin embargo, como el ejercicio pide especificar el pueblo elegido, instalamos inicialmente una central en la ra\'iz del \'arbol. Luego, por cada arista ($v_1$, $v_2$) que eliminamos instalamos una central en el pueblo representado por $v_2$.\\ Esto \'ultimo funciona porque inicialmente instalamos una central en la primera ciudad del \'arbol, con lo cual si eliminamos cualquier conexi\'on directa a \'este le ponemos la central al segundo pueblo, no al primero, que ya tiene una. Y si eliminamos una conexi\'on $v_1, v_2$ donde hay un camino del primer pueblo a $v_1$, \'este sigue recibiendo el gas del primer pueblo y $v_2$ pasa a tener una central nueva para proveer a los pueblos conectados a \'el.\\ 

\subsubsection{Implementaci\'on de Prim}
Describimos a continuaci\'on nuestra implementaci\'on del algoritmo de Prim.\\

Elegimos como ra\'iz del \'arbol generador m\'inimo al primer pueblo de la lista de pueblos de la regi\'on. Como al menos va a haber una central, instalamos una en el pueblo ra\'iz.\\
Luego realizamos un ciclo que se repite $n - 1$ veces (siendo $n$ la cantidad de nodos, es decir, ciudades), en el cual en cada iteraci\'on obtenemos el nodo m\'as cercano al \'arbol y lo agregamos al mismo.\\
Para saber cu\'al es el nodo m\'as cercano al \'arbol debemos, en cada iteraci\'on calcular la distancia m\'inima entre cada nodo que no pertenece al \'arbol y los que s\'i pertenecen.\\
Como partimos de un \'arbol que s\'olo tiene al primer nodo de la lista ($n_0$) inicialmente calculamos la distancia de cada uno de los nodos restantes a $n_0$. Luego elegimos el nodo de menor distancia y lo agregamos al \'arbol y repetimos el proceso de calcular distancias y elegir el nodo m\'as cercano.\\
As\'i ,cada vez que agregamos un nodo al \'arbol volvemos a computar la m\'inima distancia de los nodos restantes comparando la distancia que cada uno tiene guardada como m\'inima con la distancia al nuevo nodo.\\
Una vez agregados todos los nodos al \'arbol tenemos un \'arbol generador m\'inimo.\\

\newpage

\subsection{Demostraci\'on de la resoluci\'on}

Veamos que: sea T un \'arbol generador m\'inimo, eliminar las k-1 aristas de mayor peso nos da un bosque de k componentes de riesgo m\'inimo.\\

\subsubsection{Caso base - k = 1}

Queremos ver que para el caso de k=1, dado el grafo de n ciudades, cualquier AGM de ese grafo es un \'arbol generador de riesgo m\'inimo.\\
En este caso, como k=1, k-1=0 no hay que eliminar ninguna arista; basta con ver que el AGM es de riesgo m\'inimo.\\

Llamemos $T$ al AGM y supongamos que existe otro \'arbol generador $T'$ que tiene un riesgo menor que $T$. Sea $e = (n_i, n_j)$ la arista de peso m\'aximo de $T$ ($peso(e) == riesgo(T)$ \footnote{$riesgo(T) = \max_{e \in T} peso(e)$}), queda claro que \footnote{$riesgo(T') < riesgo(T) \Rightarrow \max_{e \in T'} peso(e) < \max_{e \in T} peso(e) \Rightarrow (\forall e' \in T')\, peso(e') < peso(e)$}:
\begin{equation}\label{eq}
(\forall e' \in T')\, peso(e') < peso(e)
\end{equation}
Entonces, me puedo armar un \'arbol $T''$ tal que $peso(T'') < peso(T)$ de la siguiente manera:

\begin{itemize}
 \item Elimino de $T$ la arista $e = (n_i, n_j)$. Luego, el \'arbol generador se parte en exactamente dos componentes conexas, la que posee al nodo $n_i$ y la que posee al nodo $n_j$.
 \item Sabemos que existe una arista $e'$ en $T'$ que une ambas componentes conexas, ya que $T'$ es un AG del mismo grafo del que $T$ es un AGM (con lo cual $V(T) == V(T')$ y adem\'as $T'$ es conexo por ser un AG). Por \'ultimo, notemos que $e' \neq e$ pues $peso(e') < peso(e)$ por (\ref{eq}). Entonces puedo unir ambas componentes con $e'$ y esto no me genera ciclos (ya que estoy uniendo dos componentes conexas distintas).
\end{itemize}
Notar que si hay m\'as de una arista $e$ debemos eliminarlas tambi\'en y hacer el intercambio por su correspondiente en $T'$. 
Con esto, obtenemos un \'arbol generador $T''$ tal que $peso(T'') < peso(T)$, pero esto es \textbf{absurdo} ya que estabamos suponiendo que $T$ era un AGM. El absurdo proviene de suponer que existe un AG con riesgo m\'inimo que no es AGM.\\

\subsubsection{Paso inductivo}

Suponemos ahora que vale P(k) = eliminar las k-1 aristas de mayor peso nos da un bosque de k componentes de riesgo m\'inimo, $(\forall k \leq n)$. Queremos ver que vale P(k+1).\\

\begin{itemize}
\item Caso $k + 1 \geq n$. En este caso vamos a cortar todas las aristas del \'arbol, terminando con n componentes de grado 0, es decir riesgo = 0. Es trivial ver que el riesgo en este caso es m\'inimo y que ningun otro AG lo podr\'ia mejorar.
\item Caso $k + 1 < n$. Por HI sabemos que los \'arboles formados por las k primeras eliminaciones de aristas (siempre de mayor a menor) son de riesgo m\'inimo.\\
Podemos entonces, de todos los \'arboles formados, concentrarnos en el que tiene la arista de mayor peso; que es la que se elimina en el paso k + 1. Llamemoslo $B$\\
En el caso base probamos que $T$ el \'arbol generador m\'inimo sobre el cual estamos trabajando es de riesgo m\'inimo. En particular sabemos que no hay ninguna arista de $B$ que pueda ser reemplazada por otra de menor peso.\\
Adem\'as, por HI sabemos que $B$ es de riesgo m\'inimo, ya que se obtuvo de eliminar k aristas de $T$.\\
Ahora, si eliminamos $e = (n_i, n_j)$ la arista de peso m\'aximo de $B$, obtenemos dos \'arboles, de los cuales al menos uno tiene a $e' = (n_i', n_j')$, la arista de $B$ que le sigue inmediatamente en peso a $e$ (es decir que $peso(e') \leq peso(e) = (n_i, n_j)$).\\
Es decir que pasamos de $B$ un \'arbol de riesgo $peso(e)$ a $B', B''$ dos \'arboles cuyo riesgo es, a lo sumo $peso(e)$; y el riesgo sigue siendo m\'inimo ya que es menor o igual al que ten\'iamos antes.\\ 
\end{itemize}

\subsection{Complejidad del algoritmo}

Veamos la complejidad del algoritmo propuesto utilizando un pseudoc\'odigo que facilite el an\'alisis.

\begin{itemize}
\item poner puebloNuevo $\leftarrow$ primer pueblo de la lista
\item agregar a arbolPueblos $\leftarrow$ (puebloNuevo, puebloNuevo)
\item instalar central en puebloNuevo y poner centralesInstaladas $\leftarrow$ 1
\item poner i $\leftarrow$ 0
\item mientras i < cantidadPueblos - 1 (Agregamos los pueblos uno a uno, n iteraciones)
\begin{itemize}
	\item actualizarDistancias respecto a puebloNuevo (Lo analizamos m\'as adelante, toma O(n))
	\item poner masCercano $\leftarrow$ pueblo mas cercano fuera del arbol
	\item poner cercanoEnArbol $\leftarrow$ pueblo del arbol con el que masCercano tiene distancia minima
	\item agregar arbolPueblos $\leftarrow$ (cercanoEnArbol, masCercano) (Agregar un elemento a una lista es O(1))
\end{itemize}
\item ordenar arbolPueblos de mayor a menor por distancia entre pares 
\item mientras centralesInstaladas < cantidadCentralitas (k iteraciones, que a lo sumo son n)
\begin{itemize}
	\item poner aristaMayor $\leftarrow$ primer elem de arbolPueblos
	\item instalarCentral(aristaMayor.segundo)
	\item eliminar aristaMayor de arbolPueblos (eliminar usando erase es O(1))
\end{itemize}
\end{itemize}

Como se puede ver, la complejidad est\'a dada por el primer ciclo, el cual hace n iteraciones (donde n corresponde a la cantidad de ciudades), ya que debe agregarlas una a una al \'arbol generador m\'inimo. Dentro de ese ciclo se actualizan las distancias de las ciudades al \'arbol; veremos la complejidad de hacerlo a continuaci\'on, pero podemos adelantar que realiza n iteraciones tambi\'en. Con lo cual la complejidad total del ciclo quedar\'ia en O(n).\\
Una vez obtenido el \'arbol lo ordenamos de mayor a menor, donde el par ($v_1$, $v_2$) es mayor a otro ($w_1$, $w_2$) si la distancia entre $v_1$ y $v_2$ es mayor a la distancia entre $w_1$ y $w_2$. Ordenar una lista con std::list::sort toma O(n*log(n)). \cite{sort}\\
Finalmente eliminamos los primeros k-1 (donde k es el n\'umero de centrales) pares de pueblos; es decir los pares que tienen mayor distancia. Cada eliminaci\'on toma O(1) usando std::list::erase \cite{erase}. En total toma O(k), que es a lo sumo O(n).\\

Vemos ahora el pseudoc\'odigo de actualizarDsitancias para confirmar que es O(n).\\

\begin{itemize}
\item poner puebloNuevo.distanciaAlArbol $\leftarrow$ 0
\item para cada pueblo en pueblos (iteramos todos los pueblos, O(n))
\begin{itemize}
	\item poner distanciaActual $\leftarrow$ pueblo.distanciaAlArbol
	\item si distanciaActual es mayor a 0 (el pueblo no es parte del arbol)
	\begin{itemize}
		\item poner distanciaNueva $\leftarrow$ distancia(puebloNuevo, pueblo)
		\item si distanciaNueva > distanciaActual poner pueblo.distanciaAlArbol $\leftarrow$ distanciaNueva
	\end{itemize}
\end{itemize}
\end{itemize}

En actualizarDistancias es claro que la complejidad est\'a dada por la iteraci\'on a trav\'es de las ciudades, es decir O(n). Luego, dentro del ciclo s\'olo se realizan comparaciones y asignaciones, que no afectan la complejidad significativamente.\\

Podemos decir ahora que la complejidad del algoritmo est\'a dada por el ciclo en el que se agregan nodos al \'arbol y que la misma es O($n^2$).

\newpage

\subsection{C\'odigo fuente}

\lstset{language=C++,
                basicstyle=\ttfamily\footnotesize,
                keywordstyle=\color{blue}\ttfamily,
                stringstyle=\color{red}\ttfamily,
                commentstyle=\color{green}\ttfamily,
                morecomment=[l][\color{magenta}]{\#},
                breaklines=true
}
\begin{lstlisting}

/* Constructor de region */
Region::Region(list<Pueblo*> * lista_pueblos, int centralitas){
	
	_centralitas = centralitas;
	_centrales_instaladas = 0;
	_tuberias_instaladas = 0;
	_pueblos = lista_pueblos;
	_arbol_pueblos = new list<pair<Pueblo*, Pueblo*> >();

}

void Region::resolver(){

	int cantPueblos = _pueblos->size();
	
	// Uso Prim para agregar pueblos al arbol

	// Elijo la primera ciudad de la lista como root
	Pueblo* puebloNuevo = *_pueblos->begin();
	pair<Pueblo*, Pueblo*> parPueblos = pair<Pueblo*, Pueblo*>(puebloNuevo, puebloNuevo);
	_arbol_pueblos->push_back(parPueblos);
	// Inicialmente solo el pueblo root tiene una central instalada
	puebloNuevo->instalarCentral();
	_centrales_instaladas = 1;

	// Pueblo mas cercano actual
	Pueblo * masCercano = puebloNuevo;

	// Agrego ciudades al arbol de a una - O(n)
	for(int i=0; i<cantPueblos-1 ; i++){
		
		// Actualizo las distancias al arbol y me quedo con la menor
		//cout << "nuevo_id: " << puebloNuevo->getId() << endl;
		masCercano = actualizarDistancias(puebloNuevo);

		// Agrego masCercano al arbol
		// En la proxima iteracion la distancia va a quedar en 0
		parPueblos = pair<Pueblo*, Pueblo*>(masCercano->getPuebloCercano(), masCercano);
		_arbol_pueblos->push_back(parPueblos);
		puebloNuevo = masCercano;
	}

	// Ordeno los pares de ciudades segun distancia (de mayor a menor)
	_arbol_pueblos->sort(compararDistancia);

	// Mientras que pueda instalar centrales achico el tam maximo de las tuberias
	// Es decir, genero k componentes conexas, cada una con una central

	list<pair<Pueblo*,Pueblo*> >::iterator p = _arbol_pueblos->begin();
	while (p != _arbol_pueblos->end()){
		if(_centrales_instaladas<_centralitas){
			// p representa la tuberia mas larga, la elimino e instalo una nueva central
			if(!((*p).second)->tieneCentral()){
				((*p).second)->instalarCentral();
				_centrales_instaladas+=1;
			}
			p = _arbol_pueblos->erase(p);
			
		}else{
			p++;
		}
	}
}

// Actualiza distancia minima al arbol de cada pueblo y devuelve el mas cercano
Pueblo* Region::actualizarDistancias(Pueblo* puebloNuevo){

	double distActual;
	double distNueva;
	double min = std::numeric_limits<double>::infinity();
	Pueblo* masCercano = puebloNuevo;

	// Antes de empezar actualizo la distancia del nuevo pueblo
	puebloNuevo->setDistanciaArbol(0.0);

	// Recorro todos los pueblos y actualizo sus distancias al arbol comparando con el ultimo p agregado
	for(list<Pueblo*>::iterator p = _pueblos->begin(); p != _pueblos->end(); p++){

		distActual = (**p).getDistanciaArbol();

		// Si no pertenece al arbol actualizo
		if(distActual > 0.0){

			distNueva = (**p).distancia(*puebloNuevo);
			
			if(distNueva < distActual){
				(**p).setPuebloCercano(puebloNuevo);
				(**p).setDistanciaArbol(distNueva);
			}

			distActual = (**p).getDistanciaArbol();

			// Si es el menor hasta el momento guardo la ciudad
			if(distActual < min){
				masCercano = *p;
			}
		}
	
	}

	return masCercano;
}

\end{lstlisting}

\subsection{Casos de prueba}

Como casos de prueba para el algoritmo elegimos inputs de los siguientes tipos:
\begin{itemize}
\item Caso en los que la cantidad de pueblos es igual a la cantidad de centrales.
\item Caso en los que la cantidad de pueblos es menor a la cantidad de centrales.
\item Caso en el que cada ciudad est\'a a distancia 1 de la siguiente.
\end{itemize}

Los casos utilizados se encuentran en el directorio /ej2/inputs/prueba y se puede verificar que funcionan ejectuando, desde /ej2, ./ej2 > inputs/prueba/algunCaso.txt.

\subsection{Performance}

Para el an\'alisis de performance de este ejercicio decidimos ejecutar lotes aleatorios de tests siguiendo los siguientes criterios:

\begin{itemize}
	\item Iteramos entre 50.000 y alrededor de 80.000 pueblos.
	\item En cada iteraci\'on incrementamos la cantidad de pueblos en 1.000 unidades.
	\item Generamos un input con coordenadas aleatorias de los pueblos.
	\item Para cada iteraci\'on tomamos los siguientes valores de centrales:
	\begin{itemize}
		\item Comenzamos desde 1 central hasta 100.000.
		\item Incrementamos en cada iteraci\'on la cantidad de centrales en 4.000 unidades
	\end{itemize}
\end{itemize}

Para realizar esto generamos dos scripts de bash, uno que va iterando la cantidad de pueblos y centrales y por cada caso invoca al otro script, quien se encarga de generar las coordenadas aleatorias para la cantidad de pueblos deseada.

En cuanto a las cantidades de pueblos y centrales entendimos que para evitar la aleatoriedad en los resultados deb\'iamos ejecutar una gran cantidad de casos e ir incrementando la cantidad de pueblos hasta un n\'umero considerable. 
Adem\'as, nos pareci\'o interesante pasar por distintas cantidades de centrales, empezando por una, en la cual no iban a ser necesarios recortes de conexiones entre pueblos, e ir incrementando los valores hasta llegar a cantidades similares a la de pueblos, en donde iba a ser conveniente realizar m\'as recortes y por ende la implementaci\'on iba a tener que realizar m\'as procesamiento y poder visualizar as\'i que la complejidad no se ve\'ia afectada.

\begin{center}
\begin{figure}[h!]
\includegraphics[scale=0.4]{./img/ej2_chart.png}
\caption{Tiempo transcurrido por cantidad de pueblos}
\end{figure}
\end{center}


Como podemos observar, con los datos obtenidos de las ejecuciones, se genera un gr\'afico con, a priori, un aspecto similar al de una funci\'on cuadr\'atica. 

Para poder observar este comportamiento mejor es necesario acompanarlo por la funci\'on $c*n*n$ con $c$ constante tal que permita ajustar la funci\'on cuadr\'atica al gr\'afico generado.

Para poder obtener dicha constante multiplicamos cada $n$, cantidad de pueblos, por s\'i mismo y luego dividimos el tiempo resultante por dicho valor. De esta manera obtenemos una relaci\'on entre el cuadrado de la cantidad de pueblos y el tiempo de ejecuci\'on para cada valor. \\
Finalmente, tomando un valor de $c$ apenas m\'as grande que el mayor obtenido en las anteriores operaciones, podemos aproximarnos a una constante que nos permita visualizar con mayor claridad que la complejidad solicitada no es superada y que el comportamiento se asemeja al deseado/solicitado.

Como peque\~na aclaraci\'on vale decir que los saltos que se producen en el eje x en el gr\'afico se corresponden con que ibamos iterando la cantidad de pueblos de mil en mil y es por eso que no se observan valores intermedios. 
As\'i tambi\'en, podemos observar que los distintos valores de tiempo para una misma cantidad de pueblos se corresponden con resultados para distintas cantidades de centrales.

Siguiendo por esta l\'inea de an\'alisis, entre los 55.000 y los 65.000 pueblos se puede notar una gran densidad en los valores, indicandonos que para dichas cantidades de pueblos, los tiempos de ejecuci\'on no se ven afectados por la cantidad de centrales.


\newpage
\section{Problema 3: Rompecolores}
\subsection{Descripci\'on del problema}

En este ejercicio se nos presenta el juego Saltos en la Matrix, el mismo se basa en el siguiente reglamento:

\begin{itemize}
\item Se posee un campo de juego cuadrado de N x N celdas.
\item Cada participante comienza en una posici\'on arbitraria $origen$ y debe llegar a la posici\'on $destino$.
\item Para moverse los participantes deber\'an ir saltando por las celdas de manera horizantal o vertical, no as\'i en diagonal.
\item Cada celda posee una potencia $p_{max}$ m\'axima de salto, los participantes podr\'an elegir una potencia $p$ entre 1 y $p_{max}$ para saltar hacia otra celda.
\item A modo de 'bonus' cada jugador posee $k$ unidades extra que podr\'a ir distribuyendo de la manera que desee y las mismas tienen como objetivo otorgarle a los participantes la posiblidad de realizar un salto m\'as potente de lo que la celda le proporciona en caso de creerlo conveniente. Por ejemplo, si un jugador est\'a ubicado en una celda que le posibilita saltar como mucho 2 celdas, pero el mismo considera beneficioso saltar 5, entonces el participante podr\'a utilizar 3 de sus unidades extra para realizar el salto deseado.
\end{itemize}

Como objetivo del ejercicio se nos pide presentar un algoritmo que tome los datos necesarios y devuelva como resultado una secuencia de saltos v\'alida que resuelva el problema en la menor cantidad de saltos posible.

Por otro lado el algoritmo no podr\'a tener una complejidad temporal de peor caso mayor a O($n^3 \cdot k$).

En caso de existir m\'as de una soluci\'on \'optima se podr\'a retornar cualquiera de ellas.


\subsection{Resoluci\'on y demostraci\'on}

Para la resoluci\'on del ejercicio decidimos modelar el tablero con un grafo dirigido. Una primera intuici\'on nos llev\'o a asignarle a cada casilla un nodo del grafo y las aristas que parten desde cada nodo representaban los posibles saltos entre casillas. Aunque este enfoque no es del todo incorrecto, no modela correctamente la variable de potencia adicional $k$ que uno posee a lo largo del juego. Si bien uno en una casilla de poder $p$ puede alcanzar las casillas tanto horizontales como verticales a distancia menor o igual a $p$, uno puede aumentar el rango de salto usando la potencia extra $k$. \\

Para poder representar esta informaci\'on en el grafo decidimos que los nodos ademas de representar casillas, representan a las mismas pero en distintos posibles estados. Estos estados pueden ser el no haber gastado ningun $k$ en el momento que se encuentra en esa casilla, o caso contrario cuanto fue gastado para llegar hasta esa casilla a lo largo del camino realizado.

%\begin{figure}[h]
%\begin{center}
%\includegraphics[scale=0.6]{./img/ej3_res1.png}
%\caption{Ejemplo de modelado del grafo. Cada casilla posee k+1 nodos que representan sus posibles estados. En este ejemplo el k = 2 y el poder de salto de la casilla (1,1) marcada en violeta es p = 2. Las lineas rojas representan saltos sin usar unidades extra de potencia y las lineas azules usandolas.}
%\end{center}
%\end{figure}

Si unimos adecuadamente los nodos respetando las condiciones del juego, podremos modelar correctamente el tablero con todas sus condiciones. Luego solo restar\'a buscar el camino m\'inimo entre el nodo que represente la casilla de inicio y alguno de los nodos posibles de la casilla de salida. Las condiciones a respetar son las siguientes:

\begin{itemize}
\item Desde cualquier nodo (en cualquier estado $q$ de gasto de potencias extras) uno puede saltar hasta $p$ casilleros para cualquiera de los cuatro sentidos cardinales. Por ejemplo, desde la casilla (3,6) con poder de salto 2 en un tablero de 7x7, puedo llegar hasta las casillas (3,7), (3,5) y (3,4) horizontalmente y (1,6), (2,6), (4,6) y (5,6) verticalmente. Esto hace que nuestro algoritmo cree una arista entre los nodos que representan la casilla (3,6) y todos los nodos de las casillas que mencionamos anteriormente que representen el mismo estado $q$ de la casilla de origen.
\item Desde cualquier nodo en un estado $q$ de gasto de potencias extras y siendo $k$ el m\'aximo gasto extra posible, puedo saltar $p+(k-q)$ casilleros horizontal y verticalmente. Si el salto que realice supera los $p$ casilleros, el nodo de llegada debe ser el del casillero destino pero en estado igual a $q+k\_gastado$, donde $k\_gastado$ es el poder adicional usado para llegar a esa casilla. Esto garantiza tener una "memoria" de los $k$ utilizados durante uno de los caminos posibles. Notese que en los nodos que representan casillas con $k$ agotados, las unicas aristas posibles son las que solo utilizan la potencia de la casilla y siempre van a nodos que representan casillas sin $k$ restante. No hacer esto haria que el jugador gane unidades de $k$ que no le corresponden.
\end{itemize}

Una vez armado este grafo dirigido que contiene todos los posibles caminos desde todas las casillas (respetando las reglas del juego), solo resta ver cual es el camino con menor cantidad de saltos que logra llegar a la casilla de salida. Cabe destacar que nuestro grafo no es un grafo con pesos, ya que eliminamos ese factor creando aristas con todos los posibles caminos dependiendo el $p$ de la casilla. Este camino minimo lo buscamos con el algoritmo de recorrido de grafos BFS, una de sus utilidades es buscar caminos minimos en grafos sin peso o con todos pesos iguales.


A continuaci\'on se muestran 2 im\'agenes a modo de ejemplo, un tablero, donde los n\'umeros de cada casillero representa la potencia de salto desde ese lugar, y la segunda imagen es un grafo donde se muestra como est\'an unidos los casilleros marcados en azul en la primera. La representaci\'on en el grafo se hace solo con estos tres casilleros para no hacer un grafo totalmente representativo, pero poco entendible.
Para poder interpretar el grafo se debe tener en cuenta la notaci\'on usada:
\begin{itemize}
\item Los nodos en negrita representan los diferentes estados de las casillas pintadas en azul en la primer imagen.
\item En los nodos se ve un valor ($x$,$y$) que representa la posici\'on en el tablero, y un valor $k$ que se identifica con la potencia restante en ese momento.
\item Las aristas presentan un valor $p$ mediante el cual se muestra la potencia (de la casilla) de salto usada, y un valor $k$ que representa la cantidad de unidades extra de potencia que se sumaron en dicho salto.
\end{itemize}

\begin{figure}[h]
\begin{center}
\includegraphics[scale=0.3]{./img/ej3_tablero.png}
\caption{Ejemplo de tablero de juego con k=1}
\end{center}
\end{figure}

\begin{figure}[h]
\begin{center}
\includegraphics[scale=0.3]{./img/ej3_grafo.png}
\caption{Grafo que representa los posibles saltos entre los casilleros (y estados) pintados en azul en el tablero anterior.}
\end{center}
\end{figure}

Como puede observarse, cada vez que se realiza un salto, si se uso alguna unidad de potencia extra $k$, el salto tiene como destino un nodo que cuenta con la cantidad $k$ restante ($k\_antes\_del\_salto$ - $k\_usado\_en\_el\_salto$), y este a su vez s\'olo tiene aristas hacia nodos con valores $k$ iguales o menores, ya que las unidades no podr\'ian recuperarse.
\newpage
\subsection{Complejidad del algoritmo}

Analizaremos la complejidad del algoritmo propuesto utilizando el siguiente pseudo c\'odigo como gu\'ia:

\begin{itemize}
\item crear una tabla de n*n con las potencias de las casillas, donde $n$ en es el ancho/alto del tablero. (Esto demora exactamente $n^2$ iteraciones)
\item crear un grafo dirigido $G$ de n*n*(k+1) nodos. (Usamos una de listas de adyacencia para representar el grafo. La estructura es un arreglo indexado por el n\'umero de nodo apuntando a punteros de listas. Crear el arreglo cuesta $n^2*(k+1)$ iteraciones, creando una lista vacia en cada iteraci\'on en tiempo constante. \cite{complejidad_lista_construct})
\item para cada nodo $v$ del grafo $G$ (Recorremos los nodos en $n^2*(k+1)$ iteraciones)
	\begin{itemize}
	\item para cada casilla $c$ de la fila del tablero donde se encuentra $v$ (Recorremos las n columnas del tablero, $n$ iteraciones)
		\begin{itemize}
			\item determino si debo unir el nodo $v$ con algun nodo de la casilla $c$ (Esto se realiza en tiempo constante ya que realiza operaciones matem\'aticas y en caso de tener que unirlos agrega un elemento a una lista en tiempo constante \cite{complejidad_lista_push_back})
		\end{itemize}
	\end{itemize}
	\begin{itemize}
	\item para cada casilla $c$ de la columna del tablero donde se encuentra $v$ (Recorremos las n filas del tablero, $n$ iteraciones)
		\begin{itemize}
			\item determino si debo unir el nodo $v$ con algun nodo de la casilla $c$ (Esto se realiza en tiempo constante ya que realiza operaciones matem\'aticas y en caso de tener que unirlos agrega un elemento a una lista en tiempo constante \cite{complejidad_lista_push_back})
		\end{itemize}
	\end{itemize}
\item busco el camino m\'inimo entre la casilla $inicio$ y la casilla $fin$ del grafo $G$ usando BFS ($O(n^3*k)$ (*))
\end{itemize}

(*) El algoritmo BFS tiene una complejidad de O(|V|+|E|), suponiendo el tablero mas denso posible donde se puede ir de cualquier casilla a cualquier casilla (de la misma fila o columna) tenemos $n^2*(k+1)$ nodos y $2*n$ aristas desde cada nodo, resultando $n^3*(k+1)$ aristas. |E| supera a |V| quedando una complejidad de $O(n^3*k)$ \\

Como analizamos, el algoritmo que asigna las aristas de cada uno de los $O(n^2*k)$ nodos se genera con un ciclo que anida 2 ciclos en paralelo de $n$ elementos, completandose toda esta operaci\'on en $O(n^2*k*2*n) = O(n^3*k)$. Finalmente se busca el camino m\'inimo en una complejidad del mismo orden de magnitud.

\newpage
\subsection{C\'odigo fuente}

\lstset{language=C++,
                basicstyle=\ttfamily\footnotesize,
                keywordstyle=\color{blue}\ttfamily,
                stringstyle=\color{red}\ttfamily,
                commentstyle=\color{green}\ttfamily,
                morecomment=[l][\color{magenta}]{\#},
                breaklines=true
}
\begin{lstlisting}
/*
 * Esta funcion decide dados dos nodos si hay una arista entre ellos y de que estado a que estado.
 * Si la hay, agrega la arista al grafo
*/
void unirNodos(directed_graph * grafo, int nro_nodo_fuente, int columna_nodo_fuente, int fila_nodo_fuente, int potencia_nodo_fuente, int estado_nodo_fuente, int columna_nodo_destino, int fila_nodo_destino, bool mirarFila, int tablero_casillas_por_lado, int tablero_k_inicial){
	
	int k_restantes = tablero_k_inicial-estado_nodo_fuente;
	int indice_fuente = 0;
	int indice_destino = 0;
	int nro_nodo_destino = 0;
	
	// Ajustamos los indices que nos interesan segun si estoy recorriendo fila o columna
	if(mirarFila){
		indice_fuente = columna_nodo_fuente;
		indice_destino = columna_nodo_destino;
	}else{
		indice_fuente = fila_nodo_fuente;
		indice_destino = fila_nodo_destino;
	}
	
	// Evito conectar la casilla consigo misma
	if(columna_nodo_fuente != columna_nodo_destino || fila_nodo_fuente!=fila_nodo_destino){
		
		// Verifico si el nodo de destino es alcanzable con la potencia de la casilla actual
		if( ((int) indice_fuente-potencia_nodo_fuente) <= (int) indice_destino && indice_destino <= (indice_fuente+potencia_nodo_fuente)){
			nro_nodo_destino = getNumeroNodo(tablero_casillas_por_lado, tablero_k_inicial, fila_nodo_destino, columna_nodo_destino, estado_nodo_fuente);
			// Conecto el nodo de la casilla actual en estado_nodo_actual 0 con el nodo de la casilla 
			// alcanzable en estado_nodo_actual 0 ya que no gaste ningun k para llegar ahi
			grafo->add_edge(nro_nodo_fuente,nro_nodo_destino);
		
		// Veo si puedo llegar usando k
		}else{
			int diferencia = indice_fuente-(potencia_nodo_fuente+k_restantes);
			
			// Si el poder de la casilla mas los k restantes del estado me lo permiten
			if( diferencia <= (int) indice_destino && indice_destino <= (indice_fuente+potencia_nodo_fuente+k_restantes)){
				int cuanto_k_necesito = 0;
				
				if(indice_fuente <= (int) indice_destino){
					cuanto_k_necesito = indice_destino - potencia_nodo_fuente - indice_fuente;
				}else{
					cuanto_k_necesito = indice_fuente - potencia_nodo_fuente - indice_destino;
				}
				// Lo voy a juntar con el nodo que puedo llegar pero a un estado correspondiente
				nro_nodo_destino = getNumeroNodo(tablero_casillas_por_lado, tablero_k_inicial, fila_nodo_destino, columna_nodo_destino, estado_nodo_fuente+cuanto_k_necesito);
				// Conecto el nodo de la casilla actual en estado_nodo_actual 0 con el nodo de la casilla 
				// alcanzable a su estado_nodo_actual correspondiente segun el gasto de k
				grafo->add_edge(nro_nodo_fuente,nro_nodo_destino);
			}
			
		}
	}
}

\end{lstlisting}

\newpage

\lstset{language=C++,
                basicstyle=\ttfamily\footnotesize,
                keywordstyle=\color{blue}\ttfamily,
                stringstyle=\color{red}\ttfamily,
                commentstyle=\color{green}\ttfamily,
                morecomment=[l][\color{magenta}]{\#},
                breaklines=true
}
\begin{lstlisting}

/*
 * Funcion que dada una fila, una columna y un estado de consumo de k, devuelve el numero de nodo correspondiente
 */
unsigned int getNumeroNodo(unsigned int tablero_casilleros_lado, unsigned int tablero_k, unsigned int fila, unsigned int col, unsigned int estado){
	
	unsigned int nro_nodo = (fila*tablero_casilleros_lado*(tablero_k+1) + (col*(tablero_k+1)) + estado);
	assert(0 <= nro_nodo);
	assert(nro_nodo < tablero_casilleros_lado*tablero_casilleros_lado*(tablero_k+1));
					
	return nro_nodo;
}
/*****************************/
/* CODIGO PRINCIPAL DEL MAIN */
/*****************************/

directed_graph * grafo = new directed_graph(cant_nodos);
		
unsigned int fila_nodo_fuente;
unsigned int columna_nodo_fuente;
unsigned int estado_nodo_fuente;
unsigned int potencia_nodo_fuente;

// Asigno las posibles aristas, i = todos los posibles nodos que representan tanto casillas sin gastos de k como con algun gasto
for(unsigned int nro_nodo_fuente = 0; nro_nodo_fuente<cant_nodos; nro_nodo_fuente++){
	
	// Datos de la casilla fuente
	fila_nodo_fuente = (nro_nodo_fuente / (k+1)) / n;
	columna_nodo_fuente = ((nro_nodo_fuente / (k+1)) % n);
	estado_nodo_fuente = nro_nodo_fuente % (k+1);
	potencia_nodo_fuente = (*potencias)[fila_nodo_fuente][columna_nodo_fuente];
					
	// Recorro la fila y conecto la casilla con todas las alcanzables de la misma
	for(unsigned int columna_nodo_destino=0; columna_nodo_destino<n; columna_nodo_destino++){
		
		int fila_nodo_destino = fila_nodo_fuente;
		
		// Si corresponde, se unen el nodo_fuente con el nodo_destino
		unirNodos(grafo, nro_nodo_fuente, columna_nodo_fuente, fila_nodo_fuente, potencia_nodo_fuente, estado_nodo_fuente, columna_nodo_destino, fila_nodo_destino, true, n, k);

	}
	// Recorro la columna y conecto la casilla con todas las alcanzables de la misma
	for(unsigned int fila_nodo_destino=0; fila_nodo_destino<n; fila_nodo_destino++){
		
		int columna_nodo_destino = columna_nodo_fuente;
		
		// Si corresponde, se unen el nodo_fuente con el nodo_destino
		unirNodos(grafo, nro_nodo_fuente, columna_nodo_fuente, fila_nodo_fuente, potencia_nodo_fuente, estado_nodo_fuente, columna_nodo_destino, fila_nodo_destino, false, n, k);
		
	}
}

\end{lstlisting}

\newpage
\subsection{Casos de prueba}

Incorporamos entre los archivos adjuntos, en TP:/ej3/input/, varios casos de prueba, entre ellos algunos casos borde y otros triviales para comprobar la correctitud del algoritmo.

\begin{itemize}
\item input\_corto: El tablero esta dispuesto de tal forma que desde el casillero inicial se puede llegar en dos saltos al final sin tener que usar unidades extras de potencia.
\item input\_largo: El caso opuesto al anterior, no hay unidades de potencia extra y todos los casilleros tienen el minimo de potencia. Todos los caminos posibles son igual de largos.
\item input\_camino: Para lograr tomar el camino mas corto en este tablero se debe administrar bien las potencias extras ya que a cada salto debe usarse una \'unica unidad de la misma.
\end{itemize}

\subsection{Performance}


\newpage
\begin{thebibliography}{2}

\bibitem{sort}
  C++ reference,
  \url{http://www.cplusplus.com/reference/algorithm/sort/}
  
\bibitem{upper}
  C++ reference,
  \url{http://www.cplusplus.com/reference/algorithm/upper_bound/}

  
\end{thebibliography}



\end{document}

\subsection{Descripci\'on del problema}

En este ejercicio se nos presenta el juego $Roban\'umeros$, un juego de dos jugadores que consiste en lo siguiente:

\begin{itemize}
\item se juega con cartas y cada una tiene dibujado un n\'umero entero (positivo o negativo)
\item se juega por turnos alternados entre ambos jugadores. Un jugador no puede elegir pasar, es decir que debe jugar en todos sus turnos
\item al comenzar el juego se pone sobre la mesa una secuencia de cartas boca arriba. La cantidad puede ser cualquiera
\item en su turno el jugador puede robar (es decir, sacar de la secuencia) la cantidad de cartas que quiera. Las cartas robadas deben ser adyacentes y el jugador puede elegir de qu\'e extremo sacarlas (izquierda o derecha). 
\item el juego termina cuando se terminan las cartas
\item gana el jugador que sume m\'as puntos con las cartas que rob\'o
\end{itemize}

Un ejemplo de la disposici\'on inicial de las cartas puede ser el siguiente:\\

*imagen cartas del enunciado*

En este caso, un jugador podr\'ia robar las cartas 2, -4 y 6 o -10, 9, 1, 6, pero no 2, 6, 9 ya que no son adyacentes.\\

Luego, se nos pide modelar un juego de Mingo y An\'ibal, quienes son expertos en el Roban\'umeros y juegan siempre de manera \'optima.\\
Entonces, la idea del ejercicio es generar un algoritmo que, dada una secuencia inicial de cartas simule el juego de Mingo y An\'ibal, indicando qu\'e cartas robar\'ia cada uno en su turno e informando qui\'en sale ganador.

\subsection{Resoluci\'on}

\subsection{Demostraci\'on de la resoluci\'on}

\subsection{Complejidad del algoritmo}

\subsection{C\'odigo fuente}

\subsection{Casos de prueba}

\subsection{Performance}
